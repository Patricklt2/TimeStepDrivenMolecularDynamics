\documentclass{beamer}
\usepackage[dvipsnames]{xcolor}
\usetheme{Hannover}
\usecolortheme{spruce}

\usepackage{graphicx}
\usepackage{animate}

% Datos
\title{Time Step Driven Molecular Dynamics}
\author{Grupo 5}
\institute{ITBA}
\date{} % sin fecha
% Numerar diapositivas
\setbeamertemplate{footline}[frame number]

% ---- Colores de prueba (bien contrastantes) ----
% Color base para bullets, numeración, títulos por defecto
\setbeamercolor{structure}{fg=blue}

% Sidebar: secciones
\usebeamercolor{title in sidebar}
\colorlet{SidebarTitleFG}{title in sidebar.fg}
%\setbeamercolor{section in sidebar}{fg=white, bg=red}          % sección activa -
\setbeamercolor{section in sidebar shaded}{fg=SidebarTitleFG!70!black} % secciones inactivas
\setbeamercolor{title in sidebar}{fg=SidebarTitleFG!70!black}
\setbeamercolor{author in sidebar}{fg=SidebarTitleFG!90!black}

% Sidebar: subsecciones
\usebeamercolor{subsection in sidebar}
\colorlet{SidebarSubsectionFG}{subsection in sidebar.fg}
\setbeamercolor{subsection in sidebar}{fg=white}           % subsección activa
\setbeamercolor{subsection in sidebar shaded}{}    % subsecciones inactivas

% Topbar (headline con palettes

\begin{document}

% Portada
  \begin{frame}
    \titlepage
    \begin{center}
      \small INTEGRANTES: MARTINA SCHVARTZ TALLONE, PATRICK LUCA TORLASCHI y SERGIO SMIRNOFF
    \end{center}
  \end{frame}

%-------------------- Sistema 1 --------------------
%
% Para el sistema 1) Solo deben presentarse los resultados (no incluir introducción, ni ecuaciones de
% integradores, ni implementación, ni animaciones, ni conclusiones) en la menor cantidad posible de
% diapositivas y debe ubicarse antes de la presentación del sistema

  \section{Oscilador Puntual Amortiguado}
  \subsection{Solución Analítica}
  \begin{frame}{Solución Analítica}

  \end{frame}
  \subsection{Análisis del Error}
  \begin{frame}{Error Cuadrático vs. dt}

  \end{frame}



% ---------------- Sistema 2 ----------------
  \section{Formación de Galaxias}

  \subsection{Introducción}
  \subsubsection{Sistema Real}
  \begin{frame}{Sistema Real frame}
    Texto sobre el sistema real.
  \end{frame}

  \subsubsection{Modelo Matemático}
  \begin{frame}{Modelo Matemático}
    Texto del modelo matemático.
  \end{frame}

  \subsection{Implementación}
  \begin{frame}{Implementación}
    Texto de implementación.
  \end{frame}

  \subsection{Simulaciones}
  \subsubsection{Geometría del Sistema}
  \begin{frame}{Geometría del Sistema}

  \end{frame}
  \subsubsection{Definición Observables}
  \begin{frame}{Definición Observables}

  \end{frame}

  \subsection{Resultados}
  \subsubsection{Conservación Energía}
  \begin{frame}{Conservación Energía}

  \end{frame}
  \subsubsection{Radio de Masa Media}
  \begin{frame}{Radio de Masa Media}

  \end{frame}
  \subsubsection{Colisión Cúmulos}
  \begin{frame}{Colisión de Cúmulos}

  \end{frame}

  \subsection{Conclusiones}
  \begin{frame}{Conclusiones}
    Texto de conclusiones.
  \end{frame}

% Cierre
  \begin{frame}{}
    \centering
    \Huge ¡Gracias por su atención!
  \end{frame}

\end{document}

